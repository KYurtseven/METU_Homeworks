\documentclass[12pt]{article}
\usepackage[utf8]{inputenc}
\usepackage{float}
\usepackage{amsmath}


\usepackage[hmargin=3cm,vmargin=6.0cm]{geometry}
%\topmargin=0cm
\topmargin=-2cm
\addtolength{\textheight}{6.5cm}
\addtolength{\textwidth}{2.0cm}
%\setlength{\leftmargin}{-5cm}
\setlength{\oddsidemargin}{0.0cm}
\setlength{\evensidemargin}{0.0cm}

\newcommand{\HRule}{\rule{\linewidth}{1mm}}

%misc libraries goes here
\usepackage{tikz}
\usetikzlibrary{automata,positioning}

\begin{document}

\noindent
\rule{0.5\textwidth}{.4pt}
\begin{flushright}

                                          \large \textbf{CENG 222}  \\[4mm]
                                          \large Statistical Methods for Computer Engineering \\[4mm]
                                         \normalsize      Spring '2016-2017 \\
                                            \large   Assignment 2
\end{flushright}


\section*{Student Information } 
%Write your full name and id number between the colon and newline
%Put one empty space character after colon and before newline
Full Name :  Koray Can Yurtseven\\
Id Number :  2099547\\

% Write your answers below the section tags
\section*{Answer 3.15}
\subsection*{a.}
At least one hardware failure means that $P\{X \geq 1\}$ , which is $1-P(0)$.

\begin{align*}
P\{X \geq 1\} =  1 - P_{(X,Y)}(0,0) = 1 - 0.52 =0.48
\end{align*}
\subsection*{b.}

Let $P_X$ and $P_Y$ be pmf of X and Y respectively. Adding the joint probabilities along the rows to find $P_X$ and along the coloumns to find $P_Y$, we have;

\begin{align*}
P_X(0) =  0.72 ,P_X(1) =  0.23 ,P_X(2) =  0.05 
\\and\\
P_Y(0) =  0.76 ,P_Y(1) =  0.17 ,P_Y(2) =  0.07
\end{align*}
Since 
\begin{align*}
P_{(X,Y)}(0,0) = 0.52 \ne P_X(0)P_Y(0) = (0.72)(0.76) = 0.5472
\end{align*}
we found a counterexample, so X and Y are not independent.
\section*{Answer 3.32}
Let $X$ be the number of crashed computers. This is the number of "success" (crashed computers) out of 4000 "trials" (computers), with the probability of success 1/800. Since $n$ = 4000, which is larger than 30 and $p$ = 1/800 , which is smaller than 0.05, we can use Poisson approximation to Binomial.$\newline$

$\lambda$ = np = 5. From Table A3,

\subsection*{a.}
Less than 10 computer crashed means that $P\{X < 10\}$ =$F(9)$ = 0.968
\subsection*{b.}
Exactly 10 computer crashed means that $P\{X < 10\}$ = $F(10)$ - $F(9)$ = 0.986-0.968 = 0.018
\section*{Answer 3.35}

Let $T$ be the event(thunderstorm) and $X$ be the Poisson number of accidents yesterday. By Bayes Rule, using Table A3;
\begin{align*}
P\{T | X=7\} = \frac{P\{X=7 | T\} P\{T\}}{P\{X=7 | T\}P\{T\} + P\{X=7 | T_2\}P\{T_2\}}
= \frac{(0.0901)(0.6)}{(0.0901)(0.6)+(0.0596)(0.4)} 
=0.6940
\end{align*} $\newline$
where, $\newline$ 

$P\{X=7 | T\}$ = 0.0901 Poisson(10) distribution (during a thunderstorm)

$\newline$

$P\{X=7 | T_2\}$ = 0.0596 Poisson(4) distribution (not during a thunderstorm).
\section*{Answer 4.4}
\subsection*{a.}
Find $K$ from the condition $\int f(x)dx$=1:
\begin{align*}
\int f(x)dx = \int_{0}^{10} (K-x/50)dx = Kx - \frac{x^2}{2.50}\Big|_{0}^{10} = 10K -1 = 1
\end{align*}
Solving for $K$, we get $K = 0.2$.
\subsection*{b.}
\begin{align*}
P\{X<5\} = \int_{0}^{5} (0.2-x/50)dx = 0.2x - \frac{x^2}{2.50}\Big|_{0}^{5} = 1- 0.25 = 0.75
\end{align*}
\subsection*{c.}
\begin{align*}
E(X) = \int xf(x)dx = \int_{0}^{10} x(0.2-x/50)dx = \frac{0.2 x^2}{2} - \frac{x^3}{3.50}\Big|_{0}^{10} = 10- \frac{20}{3} = 3 \frac{1}{3} years
\end{align*}
\section*{Answer 4.10}
Let $A$ be the event that the first specialist is working on the order. 
We know that $P(A)$ = 0.6 . 
Let $X$ be the amount of time (in hours) the order takes to be competed. $\newline$
$X$ given $A$ is exponential with rate 3, and $X$ given not $A$ is exponential with rate 2. $\newline$
\begin{align*}
P\{X>x | A\} = 1- P\{X \leq x| A\} = 1-(1-e^{-3x}) = e^{-3x} \\
P\{X>x | \overline{\rm A} \} = 1- P\{X \leq x| \overline{\rm A} \} = 1 - (1 - e^{-2x}) = e^{-2x}
\end{align*}
30 minutes is 0.5 hours.
$\newline$
P\{\textbf{First specialist is working on the order given order is not ready after 0.5hours} \}
\begin{align*}
&= P\{A | X>0.5\}&\\
&= \frac{P\{A \cap X>0.5\}}{P\{X>0.5\}}&\\
&=\frac{P\{X>0.5| A\}P\{A\}} {P\{A\} P\{X>0.5| A\} + P\{X>0.5| \overline{\rm A}\} P\{ \overline{\rm A}\}}&\\
&=\frac{0.6*e^{-3*0.5}}{0.6*e^{-3*0.5} + 0.4*e^{-2*0.5}}&\\
&=0.4764
\end{align*}

\end{document}




