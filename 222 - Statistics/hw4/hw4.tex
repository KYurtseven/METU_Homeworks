\documentclass[12pt]{article}
\usepackage[utf8]{inputenc}
\usepackage{float}
\usepackage{amsmath}


\usepackage[hmargin=3cm,vmargin=6.0cm]{geometry}
%\topmargin=0cm
\topmargin=-2cm
\addtolength{\textheight}{6.5cm}
\addtolength{\textwidth}{2.0cm}
%\setlength{\leftmargin}{-5cm}
\setlength{\oddsidemargin}{0.0cm}
\setlength{\evensidemargin}{0.0cm}

\newcommand{\HRule}{\rule{\linewidth}{1mm}}

%misc libraries goes here
\usepackage{tikz}
\usetikzlibrary{automata,positioning}

\begin{document}

\noindent
\HRule \\[3mm]
\begin{flushright}

                                         \LARGE \textbf{CENG 222}  \\[4mm]
                                         \Large Statistical Methods for Computer Engineering \\[4mm]
                                        \normalsize      Spring '2016-2017 \\
                                           \Large   Assignment 4 \\
                    \normalsize Deadline: May 26, 23:59 \\
                    \normalsize Submission: via COW
\end{flushright}
\HRule

\section*{Student Information }
%Write your full name and id number between the colon and newline
%Put one empty space character after colon and before newline
Full Name :  Koray Can Yurtseven\\
Id Number :  2099547\\

% Write your answers below the section tags
\section*{Answer 9.8}
\subsection*{a.}
Let $\mu$ denote the mean installation time in minutes. What we want is a 95$\%$ confidence interval for $\mu$.  The 95$\%$ confidence means that $1 - \alpha$ = $0.95$. $n = 64$, $X = 42$ min, and $\sigma = 5$. Since $\sigma$ is known, we can calculate our $Z$ interval:
\begin{align*}
X \pm Z_{\frac{\alpha}{2}} \dfrac{\sigma}{\sqrt[]{n}} = 42 \pm (1.96) \dfrac{5}{\sqrt[]{64}} = 42 \pm 1.225 = [40.775,43.225]
\end{align*}
where $Z_{0.025} = 1.96$ is obtained from the normal table (Table A4). The mean installation time is between $40.8$ and $43.2$ min, with 95$\%$ confidence.

\subsection*{b.}
We first get the $Z$ score for the two values. As $Z = (x -  \mu)\dfrac{\sqrt[]{n}}{\sigma}$ then;
\begin{align*}
x1 &= lowerbound = 40.775\\
x2 &= upperbound = 43.225\\
\mu &= mean = 40
\end{align*}
Thus, the two $Z$ scores are
\begin{align*}
z1 &= lower z score = (40.775 - 40)\dfrac{\sqrt[]{64}}{5} = 1.240\\
z2 &= upper z score = (43.225 - 40)\dfrac{\sqrt[]{64}}{5} = 5.160\\
\end{align*}
Using table, the left tailed ares between theze $Z$ scores are:
\begin{align*}
P(Z < z1) &= 0.892 \\
P(Z < z2) &= 0.999
\end{align*}
Thus the are between them, by subtracting these areas is;
\begin{align*}
P(z1 < Z < z2) = 0.107
\end{align*}	

\section*{Answer 9.16}
Let $n_1$ =250, $n_2$ = 300, $p_1 = \dfrac{10}{250} = 0.04$, and $p_2 = \dfrac{18}{300} = 0.06$.

\subsection*{a.}
A $98\%$ confidence interval for $P_1 - P_2$is
\begin{align*}
&=(p_1 - p_2) \pm z_{\frac{0.02}{2}} \sqrt[]{\dfrac{p_1(1-p_1)}{n_1} + \dfrac{p_2(1-p_2)}{n_2}}\\
&=(0.04 - 0.06) \pm z_{0.01} \sqrt[]{\dfrac{(0.04)(0.96)}{250} + \dfrac{(0.06)(0.94)}{300}}\\
&=(0.04 - 0.06) \pm 2.33 \sqrt[]{\dfrac{(0.04)(0.96)}{250} + \dfrac{(0.06)(0.94)}{300}} \\
&=(-0.0631,0.0231)
\end{align*}

\subsection*{b.}
From confidence interval, one can conclude that there is no significant difference between the quality of two sets.
\section*{Answer 10.3}

Let's first calculate our array.
\begin{align*}
(<-2.0) &--> 4\\
(-2.0,-1.5) &--> 4\\
(-1.5,-1.0) &--> 15\\
(-1.0,-0.5) &--> 9\\
(-0.5,0.0) &--> 22\\
(0.0,0.5) &--> 15\\
(0.5,1.0) &--> 12\\
(1.0,1.5) &--> 11\\
(1.5,2.0) &--> 7\\
(>2.0) &--> 1
\end{align*}
\subsection*{a.}
The corresponding standard normal probabilities and the expected number of observations with $n = 100$ are the following:
\begin{center}
\begin{tabular}{ c c c c c}
 Array & Normal Prob.& Expected Counts & Observed- Expected & Chi Value \\ 
 (Less than -2.0) & 0.023 & 2.3 & 1.7 & 1,12\\  
 (-2.0,-1.5) & 0.044 & 4.4 & -0.4 & -0,19\\  
 (-1.5,-1.0) & 0.092 & 9.2 & 5.8 & 1,91\\
 (-1.0,-0.5) & 0.150 & 15.0 & -6.0 & -1,54\\
 (-0.5,0.0) & 0.191 & 19.1 & 2.9 & 0,66\\
 (0.0,0.5) & 0.191 & 19.1 & -4.1 & -0,93\\
 (0.5,1.0) & 0.150 & 15.0 & -3.0 & 0,77\\
 (1.0,1.5) & 0.092 & 9.2 & 1.8 & 0,59\\
 (1.5,2.0) & 0.044 & 4.4 & 2.6 & 1,24\\
 (More than 2.0) & 0.023 & 2.3 & -1.3 & -0,85\\ 
\end{tabular}
\end{center}
The chi-square statistic is the sum of the squares of the values in the last column, and is equal to  12,76.\\
Data are divided into 10 bins and we estimated 2 parameters, we have 7 degrees of freedom. Since 12,76 is less than 14,1 we do not reject the null hypothesis that the data are normally distributed.
\subsection*{b.}
If the distribution is uniform, then we have;

\begin{center}
\begin{tabular}{ c c c c c}
 Array & Normal Prob.& Expected Counts & Observed- Expected\\ 
 (Less than -2.0) & 0.166 & 16.6 & -12.2\\  
 (-2.0,-1.5) & 0.083 & 8.3 & -4.3\\  
 (-1.5,-1.0) & 0.083 & 8.3 & 6.7\\
 (-1.0,-0.5) & 0.083 & 8.3 & 0.7\\
 (-0.5,0.0) & 0.083 & 8.3 & 13.7\\
 (0.0,0.5) & 0.083 & 8.3 & 6.7\\
 (0.5,1.0) & 0.083 & 8.3 & 3.7\\
 (1.0,1.5) & 0.083 & 8.3 & 2.7\\
 (1.5,2.0) & 0.083 & 8.3 & -1.3\\
 (More than 2.0) & 0.166 & 16.6 & -15.6\\ 
\end{tabular}
\end{center}
Without calculating the rest of the table and chi square we can conclude that this sample does not come from Uniform$(-3,3)$ distribution. Thus we reject the null hypothesis.
\subsection*{c.}
No. If these numbers come from both distribution, then there will be contradiction about this distribution's definition.

\end{document}