\documentclass[12pt]{article}
\usepackage[utf8]{inputenc}
\usepackage{float}
\usepackage{amsmath}


\usepackage[hmargin=3cm,vmargin=6.0cm]{geometry}
%\topmargin=0cm
\topmargin=-2cm
\addtolength{\textheight}{6.5cm}
\addtolength{\textwidth}{2.0cm}
%\setlength{\leftmargin}{-5cm}
\setlength{\oddsidemargin}{0.0cm}
\setlength{\evensidemargin}{0.0cm}

%misc libraries goes here


\begin{document}

\section*{Student Information } 
%Write your full name and id number between the colon and newline
%Put one empty space character after colon and before newline
Full Name :  Koray Can Yurtseven\\
Id Number :  2099547\\

% Write your answers below the section tags
\section*{Answer 1}

$\,$ $\,$ $\,$ $\,$Let $n$=1. Then:\\

$\quad$  
	$\displaystyle\sum_{j=1}^1$$j(j+1)(j+2)$$\ldots$$(j+k-1)$ = $\frac{1(1+1)(1+2)\ldots(1+k)}{(k+1)}$\\

$\quad$
	$1.2.3\ldots(1+k-1)$ = $\frac{1.2.3\ldots(1+k)}{(k+1)}$\\
    
$\quad$
	$k!$ = $k!$ \\
    
So, it works for $n=1$.\\

Assume, for $n=t$ holds, that is:\\

$\quad$  
	$\displaystyle\sum_{j=1}^t$$j(j+1)(j+2)$$\ldots$$(j+k-1)$ = $\frac{t(t+1)(t+2)\ldots(t+k)}{(k+1)}$\\
    
$\qquad$

We need to prove for $n=t+1$ holds.\\

$\quad$  
	$\displaystyle\sum_{j=1}^{t+1}$$j(j+1)(j+2)$$\ldots$$(j+k-1)$ = $\frac{(t+1)(t+2)(t+3)\ldots(t+1+k)}{(k+1)}$\\
    
$\quad$  
	$\displaystyle\sum_{j=1}^{t+1}$$j(j+1)(j+2)$$\ldots$$(j+k-1)$ =
    $\displaystyle\sum_{j=1}^t$$j(j+1)(j+2)$$\ldots$$(j+k-1)$ $+$ $(t+1)(t+2)\ldots(t+k)$\\
    
From our assumption:\\

$\quad$  
	$\displaystyle\sum_{j=1}^{t+1}$$j(j+1)(j+2)$$\ldots$$(j+k-1)$ =  $\frac{t(t+1)(t+2)\ldots(t+k)}{(k+1)}$ $+$ $(t+1)(t+2)\ldots(t+k)$\\
    
$\quad$     
    $\frac{t(t+1)(t+2)\ldots(t+k)}{(k+1)}$ $+$ $(t+1)(t+2)\ldots(t+k)$ = $\frac{(t+1)(t+2)(t+3)\ldots(t+1+k)}{(k+1)}$\\
    
$\quad$    
    $(t+1)(t+2)\ldots(t+k)$ = $\frac{(t+1)(t+2)(t+3)\ldots(t+1+k)}{(k+1)}$ -  $\frac{t(t+1)(t+2)\ldots(t+k)}{(k+1)}$ \\
    
$\quad$
	$(t+1)(t+2)\ldots(t+k)$ = $\frac{((t+1)(t+2)(t+3)\ldots(t+k))(t+1+k-t)}{(k+1)}$ \\
    
$\quad$  
    $(t+1)(t+2)\ldots(t+k)$ = $\frac{((t+1)(t+2)(t+3)\ldots(t+k))(1+k)}{(k+1)}$ \\
    
$\quad$  
	$(t+k)!$ = $(t+k)!$\\
    
It holds. By principle of induction, we've proved the statement.\\

\section*{Answer 2}

$\,$ $\,$ $\,$ $\,$\underline{Base Case}:\\

$\quad$
	$n=3$, $\quad$ $H_3$= 5$H_2$+ 5$H_1$ + 63$H_0$ \\
    
For given $H_0$, $H_1$ and $H_2$, we've calculated $H_3$= 103 which is smaller than $7^3$.\\
    
\,
\underline{Inductive Case}:\\

Suppose that it is true for every value of $n$ from 0 to $k$. We need to show that $H_{k+1}$ $\leq$ $7^{k+1}$.\\
    
$H_k$ = 5$H_{k-1}$+ 5$H_{k-2}$ + 63$H_{k-3}$ , where $H_{k}$ $\leq$ $7^{k}$\\

$H_{k-1}$ = 5$H_{k-2}$+ 5$H_{k-3}$ + 63$H_{k-4}$ , where $H_{k-1}$ $\leq$ $7^{k-1}$\\

$H_{k-2}$ = 5$H_{k-3}$+ 5$H_{k-4}$ + 63$H_{k-5}$ , where $H_{k-2}$ $\leq$ $7^{k-2}$\\

$H_{k+1}$ = 5$H_{k}$+ 5$H_{k-1}$ + 63$H_{k-2}$\\

From assumption 5$H_{k}$ $\leq$ 5.$7^{k}$, 5$H_{k-1}$ $\leq$ 5.$7^{k-1}$ and 63$H_{k-2}$ $\leq$ 63.$7^{k-2}$.\\

$H_{k+1}$ $\leq$ 5.$7^{k}$ + 5.$7^{k-1}$ + 63.$7^{k-2}$\\

$H_{k+1}$ $\leq$ 5.$7^{k}$ + 5.$7^{k-1}$ + 9.$7^{k-1}$\\

$H_{k+1}$ $\leq$ 5.$7^{k}$ + 14.$7^{k-1}$\\

$H_{k+1}$ $\leq$ 5.$7^{k}$+ 2.$7^{k}$\\

$H_{k+1}$ $\leq$ 7.$7^{k}$\\

$H_{k+1}$ $\leq$ $7^{k+1}$\\

Thus, we've shown that $H_{k+1}$ $\leq$ $7^{k+1}$ is true. By principle of strong induction, $H_{n}$ is true for all integers $n$ $\geq$ 0.\\

\section*{Answer 3}
\textbf{a)}\\

Since we are expected to pick at least Discrete Mathematic textbook, we can approach this question like this.\\

We have 12 textbooks and we are expected to pick 4 of them. $\binom{12}{4}$ is total way of picking 4 textbooks.\\

We have 7 Signal and Systems textbooks. So, if we pick all books from this group, we don't have any Discrete Mathematics book in our hands. $\binom{7}{4}$ is total way of picking 4 Signal and Systems textbooks.\\

If we subtract these from each other, we will be in a situation where we have at least 1 Discrete Mathematics textbook in our collection.\\

Thus $\binom{12}{4}$ - $\binom{7}{4}$ = 460 ways to make a collection.\\
\textbf{b)}\\

This collection can be done in 3 possible ways.\\

case $i$: 1 Discrete Mathematics book and 3 Signals and Systems book is chosen. This collection is equal to $\binom{5}{1}$ . $\binom{7}{3}$ because of the product rule.\\

case $ii$: 2 Discrete Mathematics book and 2 Signals and Systems book is chosen. This collection is equal to $\binom{5}{2}$ . $\binom{7}{2}$ because of the product rule.\\

case $iii$: 3 Discrete Mathematics book and 1 Signals and Systems book is chosen. This collection is equal to $\binom{5}{3}$ . $\binom{7}{1}$ because of the product rule.\\

Together, using sum rule, in $\binom{5}{1}$ . $\binom{7}{3}$ + $\binom{5}{2}$ . $\binom{7}{2}$ + $\binom{5}{3}$ . $\binom{7}{1}$ = 455 different ways we can make this collection.


\section*{Answer 4}

Let $a_n$ denote the number of different strings of length n that composed of only 2's and 3's and having even number of 3's.\\

$a_n$ = $2^{n-1}$\\

\section*{Answer 5}
First, write $a_n$ as below:\\

$x^3$ = $4x^2$ + $x$ - $4$\\

$x^3$ - $4x^2$ - $x$ + $4$ = 0\\

$(x^2-1)$$(x-4)$ = 0\\

Roots of this equation are $x = 1$, $x = -1$ and $x = 4$\\

$a_n$= a.$(1)^n$ + b.$(-1)^n$ + c.$(4)^n$\\

We know that $a_0$ = 4, $a_1$ = 8 and $a_2$ = 34\\

$a_0$ = a.$(1)^0$ + b.$(-1)^0$ + c.$(4)^0$\\

4 = $a + b + c$\\

$a_1$ = a.$(1)^1$ + b.$(-1)^1$ + c.$(4)^1$\\

8 = $a - b + 4c$\\

$a_2$ = a.$(1)^2$ + b.$(-1)^2$ + c.$(4)^2$\\

34 = $a + b + 16c$\\

Solving these equations, we get $a = 1$ , $b = 1$ and $c = 2$.\\

Thus, $a_n$ = 1.$(1)^0$ + 1.$(-1)^0$ + 2.$(4)^0$\\


\section*{Answer 6}



\end{document}

​

