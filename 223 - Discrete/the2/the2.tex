\documentclass[12pt]{article}
\usepackage[utf8]{inputenc}
\usepackage{float}
\usepackage{amsmath}


\usepackage[hmargin=3cm,vmargin=6.0cm]{geometry}
%\topmargin=0cm
\topmargin=-2cm
\addtolength{\textheight}{6.5cm}
\addtolength{\textwidth}{2.0cm}
%\setlength{\leftmargin}{-5cm}
\setlength{\oddsidemargin}{0.0cm}
\setlength{\evensidemargin}{0.0cm}

%misc libraries goes here


\begin{document}

\section*{Student Information } 
%Write your full name and id number between the colon and newline
%Put one empty space character after colon and before newline
Full Name :  Koray Can Yurtseven\\
Id Number :  2099547\\

% Write your answers below the section tags
\section*{Answer 1}
A)\\

$f_1$: A $\rightarrow$ $A_1$,
$f_2$: A $\rightarrow$ $A_2$,
$\dots$
$f_n$: A $\rightarrow$ $A_n$ \\

So; \\

$f_1^{-1}$: $A_1$ $\rightarrow$ A,
$f_2^{-1}$: $A_2$ $\rightarrow$ A 
$\dots$
$f_n^{-1}$: $A_n$ $\rightarrow$ A\\

$f_k^{-1}$$(A_k)$ =A  where k=1,2,3 $\dots$n\\

$\cap^n_{k=1}$$f_k^{-1}$$(A_k)$ =A\\

$E$ = $\Pi^n_{i=1}$ $E_i$\\

A $\subset$ E, Hence $A$ = $\Pi^n_{i=1}$ $A_i$\\

$\Pi^n_{i=1}$ $A_i$ = $A$\\

Therefore; $\Pi^n_{k=1}$ $A_k$ = $\cap^n_{k=1}$$f_k^{-1}$$(A_k)$\\
B)\\

$f_2(x)$=$x_2$	where	$f_2$: E $\rightarrow$ $E_2$\\

$f_2$ takes all x's ($x_1$,$x_2$,$x_3$,$\dots$,$x_n$) and go to $x_2$, which is an one element. For example;\\

$f_2$$(x_1)$ = $x_2$\\

$f_2$$(x_2)$ = $x_2$\\

$\dots$\\

$f_2$$(x_k)$ = $x_2$\\

So, it is not 1-1.\\
C)\\

$f_1$$(x)$: E $\rightarrow$ $E_1$\\

Since our domain is E and our range is $E_1$ all elements taken from $E_1$, there is a preimage on E.
\section*{Answer 2}
A)\\

If $f$ is 1-1, then $f$ has an inverse. For all even numbers ((2k+2), k $\in$ N) taken from N, we can find an x value such that  x $\in$ $Z^{-}$.\\

Likewise, for all odd numbers((2+1),k $\in$ N) taken from N we can find an x value such that x $\in$ $Z^{+}$$\cup$ 0.\\

Therefore,$f$ has inverse.\\
B)\\

From part A, we see that for all 2k+2,  k $\in$ N, we can find a x value such that  x$\in$ $Z^{-}$. So;\\

$f_1^{-1}(26)$= a,  a $\in$ $Z^{-}$\\

$f(a)$=26,  a $<$ 0\\

26 = $|a|$\\

$|a|$ = 13, Since a is less than 0;\\

a = -13\\

$f_1^{-1}(26)$= -13

\section*{Answer 3}


\section*{Answer 4}
If A and B are countable sets, then A $\cup$ B is also countable.

Proof: Suppose that A and B are both countable sets. Without loss of generality, we can assume that A and B are disjoint. (If they are not, we can replace B by B$\setminus$A, because\\
A $\cap$(B$\setminus$A) = $\emptyset$ and A$\cup$(B$\setminus$A) = A$\cup$B) Furthermore, without loss of generality, if one of the two sets is countably infinite and other finite, we can assume that B is the one that is finite.\\

There are three cases to consider: (i) A and B are both finite, (ii) A is infinite and B is finite, and (iii) A and B are both countable infinite.\\

Case(i): Note that when A and B are finite, A$\cup$B is also finite, and therefore, countable.\\

Case(ii): Because A is countably infinite, its elements can be listed in an infinite sequence $a_1$, $a_2$, $a_3$, $\dots$ $a_n$,$\dots$ and because B is finite, its terms can be listed as  $b_1$, $b_2$, $b_3$,$\dots$ $b_m$ for some positive integer m. We can list the elements of A$\cup$B as $b_1$, $b_2$, $b_3$,$\dots$ $b_m$,$a_1$, $a_2$, $a_3$, $\dots$ $a_n$$\dots$. This menas that A$\cup$B is countably infinite.\\

Case(iii):Because both A and B are countably infinite, we can list their elements as $a_1$, $a_2$, $a_3$, $\dots$ $a_n$$\dots$ and $b_1$, $b_2$, $b_3$,$\dots$ $b_n$ $\dots$ respectively. By alternating terms of these two sequences we can list the elements of  A$\cup$B in the infinite sequence $a_1$,$b_1$,$a_2$,$b_2$,$a_3$,$b_3$,$\dots$ $a_n$,$b_n$,$\dots$. This means A$\cup$B must be countably infinite.\\

We have completed the proof, as we have shown that A$\cup$B is countable in all three cases.\\

Assume that E$\setminus$S is countable.\\

S is given as countable.S $\cup$( E$\setminus$S) = E. By theorem above, E must be countable. E is given as uncountable. Hence, contradiction ( $\bot$).\\

Therefore,  E$\setminus$S is uncountable.\\
\section*{Answer 5}
A)\\

If; $n$ $\equiv$ $1(mod3)$\\

$3k+1=n$\\

$n(n+1)$$\equiv$$2(mod3)$\\

$(3k+1)(3k+1+1)=9k^2+6k+3k+2$\\

$9k^2$ is divisible by 3.\\

$9k$ is divisible by 3.\\

2 is not divisible by 3.\\

Hence, $n(n+1)$$\equiv$$2(mod3)$\\

Else; which is  $n$ $\not\equiv$ $1(mod3)$\\

 $n(n+1)$$\equiv$$0(mod3)$
There are two cases; (i)$n$$\equiv$ $2(mod3)$ and  (ii) $n$ $\equiv$ $0(mod3)$\\

Case(i)\\ 

$n$ $\equiv$ $2(mod3)$\\

$3k+2=n$\\

$(3k+2)(3k+2+1)=(3k+2)(3k+3) = 3(3k+2)(k+1)$ which is divisible by 3.\\

Case(ii)\\

$n$ $\equiv$ $0(mod3)$\\

$3k=n$\\

$(3k)(3k+1)=3k(3k+1)$ which is divisible by 3.\\
B)\\

gcd(123,277) = gcd(277,123)\\

gcd(123, 277mod123) = gcd(123,31) where 277mod123 = 31\\

gcd(123,31) = gcd(31,123)\\

gcd(31,123mod31) = gcd(31,30) where 123mod31 = 30\\

gcd(31,30) = gcd(30,31)\\

gcd(30,31mod30) = gcd(30,1) where 31mod30 = 1\\

gcd(30,1) = gcd(1,30)\\

gcd(1,30mod1) = gcd(1,0) where 30mod1 = 0\\

gcd(1,0) = 1. Therefore gcd(123,277) = 1\\
C)\\

Table 7 eqn1 says that $P$$\rightarrow$$Q$ $\equiv$ $\neg$$P$ $\lor$ $Q$\\

Let "p$>$2 is an even prime" be  $P$ and "p $>$ $2^{100}+1$" be $Q$\\

Since only even prime is $2$, first statement $P$ is false. By given equation given above, if first statement is false, its' negation is true and by "Or introduction" we can conclude that whole statement is correct.



\end{document}

​

